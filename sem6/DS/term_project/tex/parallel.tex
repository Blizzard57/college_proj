\chapter{Parallel RNG}
Parallel algorithms for RNGs is just a generalization or 
an extension of the sequential RNGs. The main technique 
used for such generalization is splitting.

\section{Splitting Techniques}
These algorithms work on the principle that the elements of 
sequential RNG must be distributed in some way among all 
the processors.
\\\\
These systems create a single stream of random numbers 
much more quickly than a single processor system, but as 
the underlying formula for generation of random numbers 
remain the same, the quality of random numbers is not 
expected to increase.

\subsection{Sequence Splitting}
This is the simplest, where N different processors are given 
the task to generate P random numbers, each processor giving 
disjoint sequences.
\\\\
The only possible issue with that even though the numbers 
generated are disjoint, they might still have some 
correlation and merging the sequence might make the 
quality of randomness decrease.

\subsubsection{Code}
\lstinputlisting[basicstyle=Large,style=py]{code/seq_split_copy.py}

\noindent \LaTeX is not allowing @ in the line ray.remote, 
which should be present. Working code provided in Code folder.

\subsection{Random Tree Method}
In the random tree method, two LCGs are used in congruence 
to create random sequences defined as :

\begin{align*}
    L_{k+1} &= a_L L_k \mod m\\
    R_{k+1} &= a_R R_k \mod m
\end{align*}

\noindent In this, the left ones are used to generate new 
sequences whereas the right ones are used to generate the 
random numbers.
\\\\
The benefits of the method are that new random
sequences are produced in a reproducible and noncentralized 
fashion. An example of such usefulness is when dynamic 
amount of child process require a start number to create 
a random sequence, so the parent can just pass on the 
$L_k$ as the parameter, while the child uses $R_k$ for 
computation.
\\\\
The major downside is that the quality of randomness has not 
increased, but rather now unknown correlations between 
different $R_k$ sequences can exist.

\subsubsection{Code}
\lstinputlisting[basicstyle=Large,style=py]{code/random_tree_copy.py}

\subsection{Leapfrog Method}
The leapfrog method is a variant of the Random Tree method 
specifically for systems with fixed amount of generators. 
In this, the constant for $R_k$ is tweaked in the following 
manner :
\begin{align*}
    L_{k+1} &= a_L L_k \mod m\\
    R_{k+1} &= a_L^n R_k \mod m
\end{align*}

\noindent The benefit of this method is that the sequence 
will not overlap for a $P/n$ period, where $P$ is the period 
of L sequence and n is the number of subsequences required. 
The disadvantage being that the number of subsequences 
required must be known.

\subsubsection{Code}
\lstinputlisting[basicstyle=Large,style=py]{code/leapfrog_copy.py}

\subsection{Modified Leapfrog}
One final method is known as modified leapfrog, which is a 
small modification on leapfrog in the case where the number 
of random numbers in a sequence are known but not the number 
of sequences. It is defined as follows :
\begin{align*}
    L_{k+1} &= a_R^n L_k \mod m\\
    R_{k+1} &= a_R R_k \mod m
\end{align*}

\noindent It has the same advantages and disadvantages of 
the standard leapfrog method.

\subsubsection{Code}
\lstinputlisting[basicstyle=Large,style=py]{code/mod_leapfrog_copy.py}